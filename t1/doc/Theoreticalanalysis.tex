\section{Theoretical Analysis}
\label{sec:analysis}
This section, the circuit shown in Figure 1 is analysed
theoretically, in terms of nodal and mesh method.

\subsection{Nodal method}

$R_1=1.0216234171$

$R_2=3.0213296603$

$R_4=4.17287588373$

$R_5=3.07453996538$

$R_6=2.06761158432$

$R_7=7.0023872588$

$I_d=1.00202530449$

$K_b=1.00202530449$

$K_c=8.38330387808$


The circuit has $7$ nodes,labeled as $1$ to $7$.Node $0$ is chosen as ground($V_0=0$).Using KCL for essential nodes(node 1,4,5,and 6),we have:


Node 1: $\frac{V_1}{R_7}+\frac{V_2-V_1}{R_6}$

Using conductance, $G=\frac{1}{R}$, we get:

$V_1G_7+(V_2-V_1)-G_6$.....................................................(1)

Node 4:

$(V_3-V_4)G_1-(V_4-V_7)G_3+(V_5-V_4)G_2$..................................(2)

Node 5:

$I_b=(V_5-V_4)G_2$..........................................................(3)

Node 6:

$I_d-I_b=(V_6-V_7)G_5$.......................................................(4)

For node 7, $V_7=V_c=K_cI_c$


Using octave to find the nodes voltages we get:

$V_1=C_3$\\

$V_4=\frac{G_3C_3+I_b}{G_1+G_3}$\\


$V_5=\frac{G_2G_3C_3+I_b(G_1+G_2+G_3)}{G_2(G_1+G_3)}$\\


$V_6=C_3+\frac{I_d}{G_5}$

Substituting with the known values we have:

$V_1=3.02$\\

$C_3+3.0745=V_b$\\

$V_6=6.09$\\

$0.68C_3+I_b=V_5$\\

$0.68(3.02)+I_b=V_5$\\

$V_5=2.05+I_b$\\

$0.25(C_3+I_b)=V_4$\\

$0.25(3.02+I_b)=V_4$\\

$0.76+0.76I_b=V_4$\\

$\frac{-V_5-V_4}{R_2}+I_b$\\


$\frac{-2.05-I_b0.76-0.76I_b}{2.01}=-I_b$\\


$\frac{-2.70I_b-0.76}{2.01}=-I_b$\\


$-2.70I_b-0.76=-2.01I_b$\\

$0.69I_b=0.76$\\

$I_b=-1.10mA$\\

$V_5=2.05-1.10$\\

$V_5=0.95v$\\

$V_4=-0.076v$\\

$V_4=0.076v$\\

$V_6=6.09V$\\

$V_1=3.02v$


\subsection{Mesh method}

Using KVL to find the currents in a circuit in Figure~\ref{---------},we have:

Mesh 1:

$I_1=-I_b=-K_bV_b$..............................................................(1)

Mesh 2:

$I_2=-I_d$.......................................................................(2)

Mesh 3:

$-I_3(R_7+R_6+R_4)+R_4-I_4=V_c$..................................................(3)

Mesh 4:

$I_1R_3+I_3R_4-I_4(R_1+R_3+R_4)=-V_a$

Using octave we get the values of currents as:

$I_1=-I_b$\\

$I_2=I_d$\\

$I_3=\frac{I_bR-3R_4-R_4V_a+V_c(R_1+R_3+R_4)}{R_4^2-(R_1+R_3+R_4)(R_4+R_6+R_7}$\\


$I_4=\frac{I_bR_3(R_4+R_6+R_7)+R_4V_c-V_a(R_4+R_6+R_7)}{R_4^2-(R_1+R_3+R_4)(R_4+R_6+R_7}$\\

Sustituting and doing algebra we get values I as:

$I_1=1.10mA$\\
$I_2=1mA$\\
$I_3=0.65mA$\\
$I_4=1.44mA$\\

Current that is flowing in each resister is:\\

$I_{R_1}=1.44mA$\\
$I_{R_2}=1.00mA$\\
$I_{R_3}=0.34mA$\\
$I_{R_4}=0.79mA$\\
$I_{R_5}=0.1mA$\\
$I_{R_6}=0.65mA$\\
$I_{R_7}=0.65$\\


